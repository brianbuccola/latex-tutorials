%%%%%%%%%%%%%%%%%%%%%%%%%%%%%%%%%%%%%%%%%%%%%%%
%%                                           %%
%%  Filename: latex-reference-mgmt.tex       %%
%%                                           %%
%%  Date created: 2012-08-19                 %%
%%                                           %%
%%  Authors: Brian Buccola, Alanah McKillen  %%
%%                                           %%
%%  Contact: brian.buccola@mail.mcgill.ca    %%
%%           alanah.mckillen@mail.mcgill.ca  %%
%%                                           %%
%%%%%%%%%%%%%%%%%%%%%%%%%%%%%%%%%%%%%%%%%%%%%%%



% BEGIN PREAMBLE

% Declare article class, with options
% letterpaper = default US paper size
% 12pt font size
\documentclass[letterpaper,12pt]{article}

% Customize (sub)section headings
\usepackage{sectsty}
  % bold, large section fonts
  \sectionfont{\rmfamily\mdseries\bfseries\large}

% Customize margins (1 in.)
\usepackage[margin=1in]{geometry}

% For typing example code verbatim
\usepackage{verbatim}

% Define new counter for examples
\newcounter{ex-count}
% Start counter at 1 instead of 0
\setcounter{ex-count}{1}

% Define example environment
\newenvironment{example}{%
  \quote
  \textbf{Example \arabic{ex-count}}
  \verbatim
}{%
  \endverbatim
  \endquote
  \stepcounter{ex-count}
}

% Customize itemize environment
\renewenvironment{itemize}{%
  \begin{list}{$\bullet$}{%
%   move bullet all the way to left
    \setlength{\leftmargin}{0em}
  }
}{%
  \end{list}
}

% Disable indenting (set to 0 inches)
\setlength{\parindent}{0in}

% Add space between paragraphs
\usepackage{parskip}


% BEGIN BODY

\begin{document}

% title, names, date, affiliation
\begin{center}
    {\LARGE \LaTeX{} reference management}
    \\[\baselineskip]
    Brian Buccola \& Alanah McKillen \\
    August 20, 2012 \\ McGill University
\end{center}

\vspace{\baselineskip}

This document describes how to manage and cite references in \LaTeX{}. We begin
by discussing \LaTeX{}'s native \verb-\cite- command and document--internal
method of encoding reference entries (``bibitems''). We then introduce a more
robust and flexible method: using a package (\verb-natbib-) for additional
citation commands and style options, and Bib\TeX{} for standardized, external
reference management.


\section{\LaTeX{}'s native reference management}

\LaTeX{} provides a simple method of creating and citing reference entries
(``bibitems'') on the fly, with a clean and compact look. This is useful for a
document with just a few references that you'll probably never need again, e.g.,
a class assignment, presentation, etc. Although this method is limited, its
simplicity and speed are an advantage, and it's useful to understand and tease
apart \LaTeX{}'s own reference system before diving into more advanced systems.

\LaTeX{} provides a command called \verb-\cite- for automatically citing
references, and an environment called \verb-thebibliography- for listing
references, which are denoted by the \verb-bibitem- command. The latter has the
syntax \verb-\bibitem{citekey}reference_data-, where the \emph{citekey} is
argument taken by the \verb-\cite- command.

The following is a working minimal example:\footnote{%
    %
    The \texttt{9} in the first argument to \texttt{thebibliography} specifies
    the number of reference entries you have, but in a strange way: the actual
    number (9 here) is meaningless; what matters is the number of digits (1
    here). Thus, \texttt{9} (or equivalently, \texttt{4}) means only one digit
    is required for each entry, i.e., because there are 1--9 total entries,
    whereas \texttt{99} (or equivalently, \texttt{37}) means two digits are
    required for each entry, and so on. In other words, you're specifying the
    number of places needed.
}

\begin{example}
According to \cite{chomsky1959} ...
...
\begin{thebibliography}{9}
    \bibitem{chomsky1959}
        Noam Chomsky.
        \emph{On certain formal properties of grammars}.
        1959.
        Information and Control 2, 137--167.
\end{thebibliography}
\end{example}

By default, this produces inline \emph{bracketed numerals} for citations:
``According to [1] \ldots''. You can modify a bibitem's citation style with an
option: \verb-\bibitem[Chom59]{chomsky1959}- would produce ``[Chom59]'' instead
of ``[1]'', both in the paragraph and in the reference list.

Note that the reference data for a bibitem is plain text with no structure, as
far as \LaTeX{} is concerned (linebreaks are ignored). This means (1) \LaTeX{}
cannot extract info about the author, journal, year of publication, etc., hence
why the numeric citation style is so crude, and (2) it's up to you to remember
to add all relevant data, as well as formatting, e.g., italics/emphasis.

As a solution to these difficulties, it's usually best to use a package like
\verb-natbib-, together with the Bib\TeX{} reference management system.


\section{Package help and Bib\TeX{}}

\verb-natbib- provides an array of citation commands that effectively supersede
\LaTeX{}'s \verb-\cite- command by allowing you to cite the author and date,
just the author, just the date, the date in parentheses, etc. In addition,
\verb-natbib- provides the command \verb-bibliographystyle-, which allows you to
automatically format all citations (inline and in the reference list) according
to a style guide, provided you have the \verb-.bst- style file. (Most \TeX{}
distributions come with \verb-apa- and many others.)

Of course, \verb-natbib- must somehow be able to figure out a reference's
author, date, etc. from the reference entry. While it's possible to format a
bibitem (\LaTeX{}'s native entry format) in such a way that \verb-natbib- can
parse and extract the data properly, a much easier way is to use Bib\TeX{},
which allows you to keep all your references in a single text file (\verb-.bib-
extension) organized in a standardized format.

Bib\TeX{} provides standardized organization of many different publication
types, including \verb-article-, \verb-inproceedings-, \verb-phdthesis-, and
\verb-unpublished-, each of which has its own set of required and optional
fields. A Bib\TeX{} file can be created and edited with a simple text editor;
however, an easier way is to use a helper application like BibDesk or JabRef as
a frontend to your \verb-.bib- file. These applications have all the required
and optional fields programmed into them, so that you don't have to look them
up.

The following is a minimal working example of a Bib\TeX{} file---let's call it
\verb-myrefs.bib-.

\begin{example}
@ARTICLE{chomsky1959,
  author = {Noam Chomsky},
  title = {On Certain Formal Properties of Grammars},
  journal = {Information and Control},
  year = {1959},
  volume = {2},
  pages = {137--167},
}
\end{example}

To use \verb-natbib-, simply add the \verb-natbib- package to your preamble,
specify your \verb-.bib- file, and specify the bibliography style you'd like to
use (e.g., \verb-apa-).\footnote{%
    %
    Note that there is no \texttt{.bib} extension given for the bibfile. Note
    also that, written as such, the file must be located either in the same
    directory as the file you're compiling, or in the default path searched by
    Bib\TeX{}. To be safe, you can simply include the full path name, e.g.,
    \texttt{/home/john/references/myrefs}.
}

\begin{example}
\usepackage{natbib}
...
According to \citet{chomsky1959} ... After \citeyear{chomsky1959},
\citeauthor{chomsky1959} went on to ...
...
\bibliography{myrefs}
\bibliographystyle{apa}
\end{example}

To compile your document---let's call it \verb-mydoc.tex----run the
following:\footnote{%
    %
    \texttt{pdflatex} can be substituted by \texttt{latex}. For \TeX{}shop
    users: typeset first using ``LaTeX'', then ``BibTeX'', then ``LaTeX'' twice,
    which can be found under the ``Typeset'' menu.
}

\begin{example}
pdflatex mydoc.tex
bibtex mydoc
pdflatex mydoc.tex
pdflatex mydoc.tex
\end{example}

What's going on here? Essentially: (1) the first \verb-pdflatex- compile gathers
up all the citekeys and the bibfile and writes them to \verb-mydoc.aux-; (2) the
\verb-bibtex- command reads \verb-mydoc.aux- and \verb-myrefs.bib- to determine
the reference list; (3) the second \verb-pdflatex- compile inserts the reference
list but without proper labels (you'll see things like ``According to
\textbf{??}''); (4) the final \verb-pdflatex- compile fixes all labels.

Read the Bib\TeX{} documentation for further info. For a comprehensive list of
all \verb-natbib- commands and options, check out the \verb-natbib-
documentation. For a quick and easy cheat sheet, go to:
\texttt{http://merkel.zoneo.net/Latex/natbib.php}.


\section{Tips, tricks, and caveats}

\begin{itemize}

\item Some bibliography styles decapitalize all letters other than the very
    first. For example, \verb-title = {Quantification and ACD}- gets formatted
    as \emph{Quantification and acd}. This is a design feature, not flaw: it
    leaves formatting to the style file, not the bibliography file. To hardcode
    capitalization into your bibliography entry, just surround the capitalized
    letter in curly brackets: \verb-{ACD}-, \verb-{N}ew {Y}ork-, etc.

\item You can use standard \LaTeX{} commands for formatting (certain parts of)
    reference entries. This is necessary for special characters
    (\verb-g\"{o}del- for ``G\"{o}del'') and anything you need italicized
    (\verb-An Analysis of \emph{Even}- for ``An analysis of \emph{even}'').

\item The various cite commands can take \emph{multiple} citekeys as arguments,
    separated by commas. So instead of \verb-(see \cite{ref1}; \cite{ref2}-),
    you can write \verb-(see \cite{ref1,ref2})-.

\item To add a reference to the reference list that you do not actually cite in
    the document, use \verb-\nocite{citekey}-. To add \emph{all} references from
    the specified \verb-.bib- file to the reference list, use \verb-\nocite{*}-.
    This is useful for keeping a clean--looking copy of all your references,
    testing out different bibliography styles, double--checking that your
    entries are properly formatted, etc.

\item The space between entries in the reference list can be modified with the
    \verb-\bibsep- value, e.g., add \verb-\setlength{\bibsep}{0pt}- to your
    preamble to have a single linebreak (no space) between entries.

\item You can create your own macro for possessive citations (e.g., ``In
    Chomsky's (1959) opinion'') as follows:

\begin{example}
\newcommand{\citetposs}[1]{\citeauthor{#1}'s \citeyearpar{#1}}
...
In \citetposs{chomsky1959} opinion ...
\end{example}

\item Some excellent free and open--source frontends for Bib\TeX{} are JabRef
    (Windows, Linux, Mac OS X) and BibDesk (Mac OS X). They include features
    like classifying entries by keyword, creating sub--bibliographies (e.g., for
    small projects), import/export of various formats, automatic customized
    citekey generation, searching and extracting references from databases
    (Google Scholar, arXiv, CiteseerX), etc.

\end{itemize}


\end{document}
