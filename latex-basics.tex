%%%%%%%%%%%%%%%%%%%%%%%%%%%%%%%%%%%%%%%%%%%%%%%
%%                                           %%
%%  Filename: latex-basics.tex               %%
%%                                           %%
%%  Date created: 2012-01-28                 %%
%%                                           %%
%%  Authors: Brian Buccola, Alanah McKillen  %%
%%                                           %%
%%  Contact: brian.buccola@mail.mcgill.ca    %%
%%           alanah.mckillen@mail.mcgill.ca  %%
%%                                           %%
%%%%%%%%%%%%%%%%%%%%%%%%%%%%%%%%%%%%%%%%%%%%%%%



% BEGIN PREAMBLE

% Declare article class, with options
% letterpaper = default US paper size
% 12pt font size
\documentclass[letterpaper,12pt]{article}

% Add strikeout, underline capability
% "normalem" option retains default \emph{}
\usepackage[normalem]{ulem}

% Customize (sub)section headings
\usepackage{sectsty}
  % bold, large section fonts
  \sectionfont{\rmfamily\mdseries\bfseries\large}
  % italics, normalsize subsection fonts
  \subsectionfont{\rmfamily\mdseries\itshape\normalsize}

% Customize margins (1 in.)
\usepackage[margin=1in]{geometry}

% For typing example code verbatim
\usepackage{verbatim}

% Define syntax environment
\newenvironment{syntax}{%
  \quote
  \textbf{Syntax}
  \verbatim
}{%
  \endverbatim
  \endquote
}

% Define example environment
\newenvironment{example}{%
  \quote
  \textbf{Example}
  \verbatim
}{%
  \endverbatim
  \endquote
}

% Define example output environment
\newenvironment{exoutput}{%
  \quote
  \textbf{Example output}
  \endquote
}

% Define tip command for tips
\newcommand{\tip}{\textit{Tip: }}

% Customize itemize environment
\renewenvironment{itemize}{%
  \begin{list}{$\bullet$}{%
%   move bullet all the way to left
    \setlength{\leftmargin}{0em}
  }
}{%
  \end{list}
}

% Disable indenting (set to 0 inches)
\setlength{\parindent}{0in}


% BEGIN BODY

\begin{document}

% title, names, date, affiliation
\begin{center}
{\LARGE \LaTeX{} basics}
\\[\baselineskip]
Brian Buccola \& Alanah McKillen \\
February 2, 2012 \\ McGill University
\end{center}


\section{Overview}

\LaTeX{} is a markup language (and program) for typesetting documents. It
consists of many commands and environments that specify (to the compiler) how
the output document should look. In general, commands produce or modify a small
bit of text, whereas environments modify a larger part of text.

\subsection{Document structure}

\begin{itemize}

\item Every document consists of a ``preamble'' and a ``body''.

\item The preamble includes declarations of document classes, packages,
    definitions, and other document-wide specifications.

\item The body includes the main text (title, sections, \ldots).

\begin{example}
% My first LaTeX document!

% PREAMBLE
\documentclass{article}
\usepackage{ulem}

% BODY
\begin{document}
Hello beautiful world! \sout{Goodbye cruel world!}
\end{document}
\end{example}

\begin{exoutput}
Hello beautiful world! \sout{Goodbye cruel world!}
\end{exoutput}

\end{itemize}

\subsection{Commands and environments}

\begin{itemize}

\item Every command starts with a backslash, is case-sensitive, and may take
    options (in square brackets) and arguments (in curly brackets).

\begin{syntax}
\command[option1,option2,...]{arg1}{arg2}...
\end{syntax}

\begin{example}
\textbf{This sentence will be bold.}
\end{example}

\begin{exoutput}
\textbf{This sentence will be bold.}
\end{exoutput}

\item Some commands do not require options or arguments; they simply produce
    something as is.

\begin{example}
\ldots
\end{example}

\begin{exoutput}
\ldots
\end{exoutput}

\item Every environment starts with a \verb=\begin= commad and ends with an
    \verb=\end= command, whose arguments are the environment name.

\begin{syntax}
\begin{environmentname}
Text to be modified.
\end{environmentname}
\end{syntax}

\begin{example}
\begin{enumerate}
\item First item in a numbered list.
\item Second item.
\end{enumerate}
\end{example}

\begin{exoutput}
\begin{enumerate}
\item First item in a numbered list.
\item Second item.
\end{enumerate}
\end{exoutput}

\end{itemize}

\subsection{Special characters and accents}

\begin{itemize}

\item \LaTeX{} reserves a number of characters for special purposes, such as
    curly brackets, as we've seen. The following is a more complete list.

\begin{quote}
\begin{verbatim}
# $ % ^ & _ { } ~ \
\end{verbatim}
\end{quote}

\item There are usually two ways to produce these characters: (1) use a
    backslash to ``escape out'', e.g. \verb=\#= produces \#, and (2) use the
    relevant command for that symbol, e.g.  \verb=\textbackslash= produces
    \textbackslash.

\item Accents and other diacritics are produced similarly.

\begin{syntax}
\symbol{character}
\end{syntax}

\begin{example}
\^{a} \"{a} \~{a}
\end{example}

\begin{exoutput}
\^{a} \"{a} \~{a}
\end{exoutput}

\item Beginning and ending single quotation marks are produced using \`{} (grave
    accent) and $'$ (vertical quote), respectively; double quotes are produced
    by doubling these characters, \emph{not} by using the double quotation
    character \verb="=.

\begin{example}
Bill said, ``Hello''.
\end{example}

\begin{exoutput}
Bill said, ``Hello''.
\end{exoutput}

\end{itemize}

\subsection{Whitespace and comments}

\begin{itemize}

\item Consecutive ``whitespace'' characters (spaces, tabs) are treated as one
    ``space''.

\begin{example}
The extra whitespace right here        will not show up
in the output.
\end{example}

\begin{exoutput}
The extra whitespace right here        will not show up
in the output.
\end{exoutput}

\item Similarly, single line breaks are treated as one ``space''.

\begin{example}
This
will
all be
on one line.
\end{example}

\begin{exoutput}
This
will
all be
on one line.
\end{exoutput}

\item A double linebreak indicates the start of a new paragraph; more than that
    is treated as a double linebreak.

\begin{example}
This is one paragraph that stretches stretches
stretches stretches stretches onto multiple lines.

This begins a new paragraph.
\end{example}

\begin{exoutput}
This is one paragraph that stretches stretches
stretches stretches stretches onto multiple lines.

This begins a new paragraph.
\end{exoutput}

\item You can insert your own comments by appending each comment line with
    \verb=%=; the compiler ignores anything following a \verb=%= on a given
    line.

\begin{example}
Some \textit{italic} text. % My comment about fun
                           % with italics
\end{example}

\begin{exoutput}
Some \textit{italic} text. % My comment about fun
                           % with italics
\end{exoutput}

\end{itemize}


\section{Typesetting: some specifics}

The beauty of \LaTeX{} is that, with just a handful of basic commands, you can
write a nicely structured and good-looking document without the distractions of
tweaking and fiddling that come with WYSIWYG word processors. 

\subsection{Document structuring commands}

\begin{itemize}

\item To automatically create a title page, first in the preamble, use the
    \verb=\title=, \verb=\author=, and \verb=\date= commands to define the
    title, author, and date; then use the \verb=\maketitle= command in the body
    to create a title page.

\begin{example}
\documentclass{article}
\title{Minimalist Program II: Maximize Minimalism}
\author{Noam Chomsky}
\date{\today}
\begin{document}
\maketitle
In this paper, I revise my earlier proposal \ldots
\end{document}
\end{example}

\item \tip Use the \verb=\today= command to automatically insert the day on
    which the document is compiled.

\item To add an abstract, use the \verb=abstract= environment.

\begin{example}
\begin{abstract}
This paper defends the geocentric model of the
universe.
\end{abstract}
\end{example}

\item To create section and subsection headings, use the \verb=\section= and
    \verb=\subsection= commands. They take the (sub)section name as their
    argument are numbered automatically.

\begin{syntax}
\section{Section Name}
\end{syntax}

\item \tip To exclude a (sub)section number, use \verb=\section*= and
    \verb=\subsection*= instead.

\item To add a footnote, use the \verb=\footnote= command. It takes the entire
    footnote text as its argument.

\begin{syntax}
Here's some text.\footnote{Here's a footnote!}
\end{syntax}

\end{itemize}

\subsection{Modifying text}

\begin{itemize}

\item To make text bold, italics, smallcaps, or typewriter, use the
    \verb=\textbf=, \verb=\textit=, \verb=\textsc=, and \verb=\texttt= commands.

\item Alternatively, enclose the text in curly braces, preceded by the commands
    \verb=\bfseries=, \verb=itshape=, \verb=\scshape=, or \verb=\ttfamily=,
    commands.

\begin{example}
This is some \textbf{bold}, \textit{italic},
\textsc{smallcaps}, and \texttt{typewriter} text.

This is some {\bfseries bold}, {\itshape italic},
{\scshape smallcaps}, and {\ttfamily typewriter} text.
\end{example}

\begin{exoutput}
This is some \textbf{bold}, \textit{italic},
\textsc{smallcaps}, and \texttt{typewriter} text.

This is some {\bfseries bold}, {\itshape italic},
{\scshape smallcaps}, and {\ttfamily typewriter} text.
\end{exoutput}

\item \tip Some of the fullform commands can be nested:
    \verb=\textbf{\textit{hi!}}= produces \textbf{\textit{hi!}}.

\item To change font size, use the commands \verb=\tiny=, \verb=\small=,
    \verb=\large=, \verb=\Large=, \verb=\LARGE=, etc. (See documentation for
    full list.)

\begin{example}
Some {\tiny tiny}, {\small small}, normal, {\large
large}, {\Large larger}, {\LARGE even larger}
text.
\end{example}

\begin{exoutput}
Some {\tiny tiny}, {\small small}, normal, {\large
large}, {\Large larger}, {\LARGE even larger}
text.
\end{exoutput}

\item These font sizes are relative to the font size option declared in the
    document class, e.g. \verb=\large= in a \verb=12pt= document is roughly the
    same size as \verb=\Large= in a \verb=10pt= document.

\end{itemize}

\subsection{Math mode}

\begin{itemize}

\item To produce math symbols, enclose your math text in dollar symbols; this is
    called math mode.

\begin{example}
$\forall x \in \mathcal{P}(A): \exists y \in
\phi(B \times C): y = x^5 - \alpha_3$
\end{example}

\begin{exoutput}
$\forall x \in \mathcal{P}(A): \exists y \in
\phi(B \times C): y = x^5 - \alpha_3$
\end{exoutput}

\item Math symbols can be thrown right into normal text.

\begin{example}
If $\alpha$ is of type $a \rightarrow b$ \ldots
\end{example}

\begin{exoutput}
If $\alpha$ is of type $a \rightarrow b$ \ldots
\end{exoutput}

\end{itemize}

\subsection{Some useful commands}

\begin{list}{}{}
\item \verb=\ldots  = create ellipsis dots
\item \verb=\noindent  = disable indent for this paragraph
\item \verb=\hfill  = fill horizontal space (moves the following text all the
    way rightward)
\end{list}

\subsection{Some useful environments}

\begin{itemize}

\item Here's a list of some useful environments, all of which follow the syntax
    described earlier.

\begin{list}{}{}
\item \verb=itemize  = create bulleted lists
\item \verb=enumerate  = create numbered lists
\item \verb=center  = center a block of text
\item \verb=flushleft  = left-align a block of text
\item \verb=flushright  = right-align a block of text\footnote{Text is left-
        \textit{and} right-aligned (justified) by default.}
\item \verb=tabular  = create a table
\item \verb=figure  = create a figure
\item \verb=quote  = create a block quote
\end{list}

\end{itemize}


\section{Next time?}

\begin{list}{$\bullet$}{Advanced topics}
\item bibliography (Bib\TeX{})
\item labels and references
\item defining new commands and environments; redefining current ones
\item hyperlinks, PDF metadata (\verb=hyperref=)
\item presentation slides (\verb=beamer=)
\item special headers and footers (\verb=fancyhdr=)
\item custom sections (\verb=sectsty=)
\end{list}

\begin{list}{$\bullet$}{Linguistics-specific topics}
\item numbered examples (\verb=linguex=)
\item syntax trees (\verb=qtree=)
\item IPA (\verb=tipa=)
\end{list}


\end{document}
