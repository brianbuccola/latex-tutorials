%%%%%%%%%%%%%%%%%%%%%%%%%%%%%%%%%%%%%%%%%%%%%%%
%%                                           %%
%%  Filename: latex-getting-started.tex      %%
%%                                           %%
%%  Date created: 2011-12-19                 %%
%%                                           %%
%%  Authors: Brian Buccola, Alanah McKillen  %%
%%                                           %%
%%  Contact: brian.buccola@mail.mcgill.ca    %%
%%           alanah.mckillen@mail.mcgill.ca  %%
%%                                           %%
%%%%%%%%%%%%%%%%%%%%%%%%%%%%%%%%%%%%%%%%%%%%%%%

% The percent sign is for commenting.
% Comments are not read by the compiler.


% This command starts every .tex file.
\documentclass{article}


% This package properly encodes output fonts.
% Important for copying/pasting text from PDF outputs.
\usepackage[T1]{fontenc}

% This package makes links clickable.
\usepackage{hyperref}


% Commands for the title page
\title{Getting started with \LaTeX{}}       % Defines the title
\author{Brian Buccola \\ Alanah McKillen}   % Defines the author(s)
\date{December 19, 2011}                    % Defines the date

% New, simple command for inline notes
\newcommand{\note}{\textbf{Note: }}


% This command begins the document environment
\begin{document}

% This command produces a title from the commands above.
\maketitle


% The \section command begins a section.
\section{Introduction}


% The \subsection command begins a subsection.
\subsection{What are \TeX{} and \LaTeX{}?}

\TeX{} is a powerful typesetting system designed by mathematician and computer
scientist Donald Knuth back in 1978. It understands a few hundred low-level
commands called primitives, which are rarely used directly anymore. Nowadays,
most users use \LaTeX{}, which is a markup language for \TeX{} developed in the
1980s by Leslie Lamport. It incorporates many more useful and intuitive commands
that allow users to concentrate on content rather than on formatting: you write,
and \LaTeX{} does the rest.

\LaTeX{} comes in a variety of ``distributions'', the most common ones currently
being \TeX{} Live 2011 for Unix machines, and MiK\TeX{} for Windows. (The most
popular distribution for Mac OS X, Mac\TeX, is a bundle of \TeX{} Live 2011 and
some extra stuff.)

\subsection{About this document}

This document includes basic instructions on how to install the latest version
of \LaTeX{} onto Mac, Windows, and Linux machines. For more detailed
instructions specific to your operating system, just type \texttt{latex
installation}, followed by the name of your OS, into any good search engine.

In addition, this document describes the basic steps you'll run through to
write, compile, and view your first, very simple document. You may also wish to
look at the source file for this document, \texttt{latex-getting-started.tex},
to get an idea of what some of the commands and environments look like. (You can
even edit and/or compile the source file yourself for practice.)

Finally, there are some resources at the end of the document. However, as
always, a good search engine is your best resource.

If you have any questions or comments, just contact us at:

% The quote environment is an easy way to indent a block of text.
\begin{quote}
% Use \\ to begin a new line.
\href{mailto:brian.buccola@mail.mcgill.ca}{\tt brian.buccola@mail.mcgill.ca} \\
\href{mailto:alanah.mckillen@mail.mcgill.ca}{\tt alanah.mckillen@mail.mcgill.ca}
\end{quote}

Happy holidays, and happy \LaTeX{}ing!


\section{Mac OS X}

Mac users will want to download Mac\TeX{}, available at
\url{http://www.tug.org/mactex/2011/}. Mac\TeX{} is a package that consists of
the complete \TeX{} Live 2011 distribution of \LaTeX{}, as well as many useful
``\TeX{}tras'', like BibDesk (for managing references) and \TeX{}Shop (for easy
editing, compiling, and viewing of documents).

To install, click the link above, and download the Mac\TeX{} package:

\begin{quote}
\begin{verbatim}
MacTeX.mpkg.zip
\end{verbatim}
\end{quote}

Once the \texttt{.zip} file is downloaded, unzip it, double click on
\texttt{MacTeX.mpkg}, and follow the installation instructions that come up on
the screen. After installing, you will have a \texttt{tex} folder in your
applications folder which includes BibDesk, \TeX{}Shop, etc.

The package is over 1 GB in size, but it's worth the download: it contains just
about every package, macro, font, etc. that you'll probably ever need, meaning
you'll rarely (if ever) have to manually download new packages.


\section{Windows}

Windows users seem to have two options: MiK\TeX{}, the standard Windows
implementation of \TeX{} for some time now, and pro\TeX{}t, a relatively new
distribution based on MiK\TeX{}. The \LaTeX{} project website,
\url{www.latex-project.org}, recommends pro\TeX{}t, but you may want to research
a bit before choosing.

To install pro\TeX{}t, go to \url{www.tug.org/protext/}, and follow the
directions.

To install MiK\TeX{}, go to \url{www.miktex.org/2.9/setup}, and follow the
directions.

Since Alanah uses Mac OS X and Brian uses Linux, neither of us has any
experience installing or using \LaTeX{} on a Windows machine. However, if you
have questions, please contact us anyway, and we'll do our best to help.


\section{Linux}

The easiest way for Linux users to get \LaTeX{} is to simply check your usual
software source for \TeX{} Live. You can either install \texttt{texlive}, which
is a basic subset of the full package, or \texttt{texlive-full}, which is the
complete package. \texttt{texlive-full} is recommended, if you have the time and
storage space, since it will minimize the amount of packages, fonts, etc. you
may have to install later.

The other option is to get \TeX{} Live directly from \url{www.tug.org/texlive/}.
Just follow the directions there, and/or ask Brian for help.

\note Some Linux distros may be behind in their \TeX{} packages. For example,
Ubuntu currently ships with \TeX{} Live 2009. So check before downloading from
your software source, or get the latest version directly from the site just to
be safe.


\section{Writing, compiling, \& viewing}

Now that you've (hopefully) got \LaTeX{} installed, let's produce a document!
The basic steps are the following.

% The enumerate environment begins a numbered list.
\begin{enumerate}
\item Using a good text editor, open a new text file, give it a \texttt{.tex}
    extension, and write up your document in plain text, using \LaTeX{} markup
    (commands) for typesetting.
\item Save the file, and compile it using \LaTeX{}, which produces a PDF output
    (or possibly DVI, depending on which command/setting is used).
\item View your output to see how it looks.
\end{enumerate}

\note By ``good text editor'', we just mean a text editor with \LaTeX{} syntax
highlighting. The editors that come with \TeX{}Shop (bundled with Mac\TeX{}) and
with \TeX{}nicCenter (bundled with pro\TeX{}) have syntax highlighting. Linux
users have several options available, but we recommend Vim.

\note \TeX{}Shop (and probably \TeX{}nicCenter) has a button called ``typeset''
which, when pressed, saves your file, compiles it, and opens the output in a
document viewer, all at once.

We'll now go through these steps in more detail, one by one.

\subsection{Step 1: Write}

Using your editor, open up a blank text file, and give it a \texttt{.tex}
extension by saving the file as \texttt{hello-world.tex}. Now your editor knows
to look out for \LaTeX{} markup.

The first line you should write is the following. \note \LaTeX{} commands always
start with a backslash.

\begin{quote}
\begin{verbatim}
\documentclass{article}
\end{verbatim}
\end{quote}

This command tells \LaTeX{} what class of document you wish to write. The most
common by far is the \texttt{article} class; it's suitable for nearly
everything, including research papers, handouts, etc.

Next, begin the actual document by writing the following command beneath the one
you just wrote.

\begin{quote}
\begin{verbatim}
\begin{document}
\end{verbatim}
\end{quote}

This command tells \LaTeX{} where the body of the document actually begins.
Everything after this is what gets produced in the final output.

Now let's produce a sentence. Beneath \verb=\begin{document}=, write the
following sentence.

\begin{quote}
\begin{verbatim}
Hello world!
\end{verbatim}
\end{quote}

For now, don't add anything else, because our purpose here is just to produce a
successful output.

Now we end the document by adding the following, final line.

\begin{quote}
\begin{verbatim}
\end{document}
\end{verbatim}
\end{quote}

So your entire text file should look something like this:

\begin{quote}
\begin{verbatim}
\documentclass{article}
\begin{document}
Hello world!
\end{document}
\end{verbatim}
\end{quote}

Pretty simple! Now save the file, and we are ready to compile.

\subsection{Step 2: Compile}

Compiling a \TeX{} file just means feeding \LaTeX{} a text file with a
\texttt{.tex} extension, containing only plain text and \LaTeX{} markup, and
getting a PDF (or DVI) output.

To compile, Mac and Windows users should look for a ``typeset'' or ``compile''
or ``build'' button in their text editor window. (You may first need to
configure which output file type you want.) Linux users can run the following
command in terminal:

\begin{quote}
\begin{verbatim}
pdflatex hello-world.tex
\end{verbatim}
\end{quote}

If all goes well, you should end up with a file called \texttt{hello-world.pdf},
or \texttt{hello-world.dvi}. If, however, you get an error, check your text file
for typos, and also make sure it's been saved before compiling.

\note You'll also end up with some files having extensions like \texttt{.aux},
\texttt{.out}, and \texttt{.log}. These are where \LaTeX{} keeps track of things
like labels and compiling errors. You can ignore these files for now.

\subsection{Step 3: View the output}

This the easy part. Using your favorite document viewer (or whatever came with
your \LaTeX{} distribution), open your PDF and make sure you see ``Hello
world!''.

At this point, you can go back to your source file, \texttt{hello-world.tex},
and play around, edit, re-compile, and re-view. Don't forget to save each time
before compiling!


\section{Resources}

The following is a list of useful resources for getting started with \LaTeX{}.
You can find many more by searching online for \texttt{latex resources}, and you
can almost always find help with specific topics by searching directly for
whatever it is you need, e.g. \texttt{latex tables}, \texttt{latex line
spacing}, etc.

% The itemize environment begins an unnumbered list (with bullets by default).
\begin{itemize}
\item \LaTeX{} Wikibook \\ \url{http://en.wikibooks.org/wiki/LaTeX} \\ (Perhaps
    the best place to start; it's also a great reference, and the one that
    usually comes up first in Google results.)
\item \emph{The Not So Short Introduction to \LaTeXe} \\
    \url{http://tobi.oetiker.ch/lshort/lshort.pdf} \\ (Much more detailed; super
    useful, especially when you're offline.)
\item \LaTeX{} for linguists \\
    \url{http://www.essex.ac.uk/linguistics/external/clmt/latex4ling/} \\ (Tips
    on numbered examples, syntax trees, OT tableaux, etc. \\ Caveat: some of
    this info is outdated.)
\end{itemize}


% This command ends every .tex file.
\end{document}
