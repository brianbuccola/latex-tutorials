%%%%%%%%%%%%%%%%%%%%%%%%%%%%%%%%%%%%%%%%%%%%%%%
%%                                           %%
%%  Filename: latex-trees.tex                %%
%%                                           %%
%%  Date created: 2012-08-19                 %%
%%                                           %%
%%  Authors: Alanah McKillen, Brian Buccola  %%
%%                                           %%
%%  Contact: alanah.mckillen@mail.mcgill.ca  %%
%%           brian.buccola@mail.mcgill.ca    %%
%%                                           %%
%%%%%%%%%%%%%%%%%%%%%%%%%%%%%%%%%%%%%%%%%%%%%%%


\documentclass[oneside,12pt]{article}
\usepackage[T1]{fontenc}

\usepackage[]{qtree} % options: center, nocenter
\usepackage{tree-dvips} % needed for drawing arrows

\usepackage{linguex} % needed for numbered examples


%===============

\begin{document}

% Drawing trees with qtree is possible using pdflatex (Mac users: usually
% selected by default under the typesetting menu in TexShop), but in order to
% draw arrows using the tree-dvips package, the .tex file needs to by typeset
% with latex (TeX and DVI option).

\begin{center}
{\LARGE Trees and numbered examples in \LaTeX{}}
\\[\baselineskip]
Alanah McKillen \& Brian Buccola \\
August 20, 2012 \\
McGill University
\end{center}


\section{Basic Tree Drawing}

\Tree [.A B C D E F ]
% each maximal and intermediate node is specified with a period after the
% bracket, terminal nodes have no need for square brackets


\Tree [.A B [.C D  [.E F G ] ] ]
% (right branching) Sensitive to spacing before and after brackets. Include a
% space after a node is specified, between brackets, between terminal node and
% closing bracket


\Tree [.A [.B [.C D E ] F ] G  ]
% (left branching)


\Tree [.S John [.VP kissed Mary ] ]
% simple tree, terminal nodes with more than one element must be enclosed in {},
% e.g. {the dog}


\Tree [.IP [.NP [.N\1 [.N\0 John ] ] ] [.I\1 [.I\0 +pst ] [.VP [.V\1 [.V\0 kissed ] [.NP [.N\1 [.N\0 Mary ] ] ] ] ] ] ]
% simple tree using X-bar structure. bar levels \1, head levels \0


\Tree [.A leaf_1 [.B leaf_2 [.C leaf_3  [.D leaf_4 [.E leaf_5 [.F leaf_6 [.G leaf_7 {last} ] ] ] ] ] ] ]
% use !{\qbalance} after the last node to get balanced trees, !\qsetw{2cm}
% between any two nodes to change the distance between them


\Tree [.IP \qroof{John}.NP [.I\1 [.I\0 +pst ] [.VP [.V\1 [.V\0 kissed ] \qroof{Mary}.NP ] ] ] ]
% triangle roofs \qroof which draws a "roof" above a phrase. If there is a lot
% of content under a roof it can be put on multiple lines by using \\ {the cat
% in the hat \\ with the bright \\ red bow} 


\section{Drawing Arrows}

\Tree [.IP [.NP [.N\1 [.N\0 \node{M}{Mary_i} ] ] ] [.I\1 [.I\0 was ] [.VP [.V\1 [.V\0 kissed ]  \node{t}{t_i}  ] ] ] ]
% node command takes two arguments, one is the name of your node, the other
% specified what the node is
\abarnodeconnect[-1cm]{t}{M}
% \anodecurve[b]{t}[b]{M}{2.5cm}
% options specify the position the arrow stops or starts at: bl = bottom left,
% tl = top left, br, b, t, etc. Arrows can be dashed using the \makedash{}
% command placed at the beginning of the arrow command:
% {\makedash{4pt}\anodecurve...}


{\tiny \Tree [.TP \qroof{John}.DP$_j$ [.T\1 was [.VP willing [.TP [.\node{2}{DP$_2$} every [.NP book [.CP \node{1}{Op$_1$} [.C\1 that [.TP \qroof{Mary}.DP [.T\1 did [.VP read \node{t1}{t$_1$} !{\qbalance} ] ] ] ] ] ] ] !\qsetw{5cm} [.TP PRO$_j$ [.T\1 to [.VP read \node{t2}{t$_2$} !{\qbalance} ] ] ] ] ] ] ]}
\anodecurve[b]{t1}[b]{1}{2.5cm}
\anodecurve[r]{t2}[t]{2}{5.0cm}
% adjusting the size of the tree using standard latex font size commands:
% {\tiny} {\small}, etc.

\vspace{2cm}


\section{Trees and numbered examples}

\ex. \Tree [.A [.B C D ] [.E F ]  ]
% numbering using linguex package, tree can be not centered using
% \qtreecenterfalse before the \Tree command


\ex.
\a. \Tree [.S John [.VP kissed Mary ] ] 
\b. \Tree [.IP [.NP [.N\1 [.N\0 John ] ] ] [.I\1 [.I\0 +pst ] [.VP [.V\1 [.V\0 kissed ] [.NP [.N\1 [.N\0 Mary ] ] ] ] ] ] ]
% multiple trees per example vertical


\ex.
a. \Tree [.S John [.VP kissed Mary ] ]
\hskip 2cm
b. \Tree [.IP [.NP [.N\1 [.N\0 John ] ] ] [.I\1 [.I\0 +pst ] [.VP [.V\1 [.V\0 kissed ] [.NP [.N\1 [.N\0 Mary ] ] ] ] ] ] ]
% multiple trees per example horizontal


\end{document}
